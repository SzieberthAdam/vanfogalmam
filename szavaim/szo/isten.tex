\startcomponent isten
\product szavak
\startSzo[title=Isten,reference=szo:Isten]
\startBek[mindenhatosagi_bev]
Isten megragadásához a következő állításból indulok ki:
\stopBek

\startAxioma[title=Mindenhatósági]
Isten mindenható.
\stopAxioma

\startBek[mindenhatosagi_magyarazat]
Mindenhatóság alatt tényleg \emph{mindent} értek.
A mindenható minden szférában -- lelki, szellemi, anyagi és minden szinteken -- mindenre hat, kivétel nélkül. \lasd{lélek} \lasd{szellem} \lasd{anyag}
\stopBek

\startBek[mindenhatosagi_formak]
A mindenhatósági axiómában az egyes számban mondott nagybetűs Isten az egységet és a személyes aspektust képviseli, a mindenhatóság pedig a végtelenséget és a személytelenséget.
Isten tehát egyszerre egy és végtelen, egyszerre személyes és személytelen.
A személyes Isten az Isten, a személytelen Isten pedig az Őserő.
\stopBek

\startAxioma[title=Hatás]
Isten hatása az éltetés.
A teremtés az éltetés kezdete, a hullajtás pedig a vége.
\stopAxioma

\startBek[hatas_magyarazat]
Isten mindennek a végső fenntartója és éltetője, az életerő belőle árad.
Isten az eredő és a nyelő, a teremtő és a pusztító hatalom.
Eredő és nyelő, kilégzés és belégzés, ez a világ lüktetése.
\stopBek

\startBek[rekurziv1]
Minden közeg -- a fizikai sík is -- Isten teremtménye, amit Isten ereje tart fenn és fog elhullajtani később.
Minden rész ugyanígy.
Minden Istenben van és Isten is mindenben benne van.
Minden egészben van rész és minden részben ott az egész.
Ebből következik a létező dolgok rekurzivitása. \lasd{rekurzivitás}
\stopBek

\startBek[teremto_es_hullajto_ciklus]
Az itt hullajtott az oda emelt az ott hullajtott az ide süllyesztett.
Ittet és ottot Isten élteti:
Istenben terem hát minden, ami Istenben, Isten által hullajtott el.
\stopBek

\startBek[teremto_es_hullajto_ero_minosegek]
Teremtés és hullajtás ellentétes minőségek.
Semleges viszont a kettejük váltakozásából keletkező lüktetés, és semleges a mindent átható Őserő is.
\stopBek

\startBek[mindenhatosagi_kov_teremto_szemelytelen]
A mindenség az Őserőből formálódott.
Az Őserő az áradó erő, amiből minden más keletkezik.
Az Őserő mindenben megvan, és ahol az Őserő, ott van az Isten is.
\stopBek

\startBek[mindenhatosagi_kov_elteto_szemelytelen]
Az Őserő a fenntartója mindennek, az Őserő energiája élteti a mindenséget.
Az Őserő tökéletesen semleges és semlegesen tökéletes.
Ilyen a természet, a teremtő Isten munkája.
\stopBek

\startBek[mindenhatosagi_kov_hullajto]
Az Atya a Halál.
Ő a legbölcsebb kaszás: mindig akkor arat, amikor eljött az ideje.
Az Őserő lüktetése az élet, és e lüktetésben a kiáradást visszatérés kell kövesse.
\stopBek

\startBek[rekurziv2]
Ha Isten mindenre hat, akkor hat a saját személyes és személytelen formájára is.
A személyes Isten tehát mindentudó, az Őserő pedig végtelenül rekurzív.
\stopBek

\startBek[istentelenseg_mint_isten_resze]
Isten része az istentelenség is: Isten képes úgy kivonulni, hogy közben mégis ottmarad.
\stopBek

\startBek[mindenhatosagi_kov_sors]
Minden hatás Istené hát.
Isten szabja az emberi sorsot.
Azt kérded, hol a szabadság?
Mondom, ahol kiszabatott.
\stopBek

\startBek[rekurzivitas_kov_teremtmeny_hat]
A rekurzivitás következtében minden teremtmény végesen bír Isten jellemzőivel és hatásaival, a teremtő, az éltető, és a hullajtó erőkkel.
\stopBek

\startBek[teremtmeny_mint_isteni_helytarto]
A teremtmény tehát korlátok között hat.
Az Isteni hatás viszont e korlátok között általa érvényesül.
Így valójában a teremtmény Isten képviselője.
A teremtmény szabadságának foka a korlátok tágassága.
\stopBek

\startBek[szemelyes_isten_megragadasa]
A személyes Istent az emberi személyiségen keresztül lehet megragadni.
Ha ugyanis azon kívül keressük, az Őserőet találjuk.
Ha viszont az ember magának személyiséget tulajdonít, akkor a rekurzivitás értelmében az őt teremtő Istennek is kell legyen személyisége.
\stopBek

\startBek[isten_szemelyisege_tokeletes_es_allando]
Az Isten személyisége tökéletes és állandó, az Őserőnek tökéletesen megfelelő.
Isten önnön személyisége által jelöli ki az ember számára a végső célt.
\stopBek

\startBek[valtozo_szemelyiseg_problema]
Ha Isten személyisége változó lenne, az csak a tökéletesség felé vezető út járásának példázata okán történhetne.
Ehhez Istennek bűnöket kellene elkövetnie és meg kellene játszania magát.
Ráadásul ahány ember, annyi út, ezért az ilyesfajta példázat nem illene senkihez.
\stopBek

\startBek[emberi_fogalmak]
Az Ember Isten képviselője a fogalmak világában; alkot, megtart, és elfelejt fogalmakat.
E minőségében az ellentétpárokat is Ő maga alkotta a világ lüktetésének leírására, noha a világlüktetés maga egy semleges folyamat.
Lett sötét és lett világos, lett hideg és lett meleg.
További absztrakció kellett ahhoz, hogy az ember a két pólushoz végső minőségeket rendeljen: megszületett a jó és a rossz.
Vajon van jó és rossz az ember fogalmán kívül is?
\stopBek

\startBek[emberi_szemelyiseg]
Az emberi személyiség az ellentétpárok hatása alatt áll, pozitív és negatív töltést vesz fel, változó de megállapodhat.
\stopBek

\startBek[allando_szemelyes_isten]
A személyes Isten, ha állandó, akkor az Ő székhelye a mindenség egy állandó helyén kell legyen.
\stopBek

\startBek[origo]
Egy nyugvópont bizonyos, ez pedig az origó, az ősrend közepe, a semlegességi pont, a semmi.
Az origóban az ellentétpárok nem érvényesülnek, a személyiségnek nincs töltése.
Az origóban tökéletes az egyensúly.
\stopBek

\startBek[origo_problema]
Az origóban székelő személyes Isten abszolút, megragadása viszont nem könnyű.
Ő ugyanis sem jó, sem rossz; sem igazságos, sem igazságtalan; sem bölcs, sem ostoba; sem szerető, sem gyűlölő; sem szép, sem csúnya; sem tiszta, sem mocskos.
Fogalmaink túlnyomó része nem képes leírni Őt aki így nem is tűnik személyesnek.
Úgy tűnik, hogy az Ember személyes Istene nem az origóban van.
\stopBek

\startBek[egyseg]
Következő nyugvópontja a mindenségnek az egység.
Az egység mindennek az mértéke, azaz például minden jó egységnyi jókból kell álljon.
Egy egység jó tehát a legkisebb jó.
Egységből azonban kettő van: pozitív és negatív.
\stopBek

\startBek[egyseg_problema]
Ha a személyes Isten az egységben székelne akkor az ember könnyedén túltehetne rajta jóságban, bölcsességben és sok minden másban is.
Mégis, a rekurzivitás alapján Isten mindenben benne van, tehát az egység Isten székhelye kell legyen. %\resz{rekurziv1}
Eddig az isteni egység csak az Őserőre vonatkozott, de most látható, hogy Isten személyisége is egysége az összes személyiségeknek.
\stopBek

\startBek[vegtelen]
Végső nyugvópontja a mindenségnek a végtelen.
Ember számára elképzelhetetlen, Isten számára viszont szilárd pont.
A végtelen a lehetséges legnagyobb, azaz például minden jó a végtelen jó része kell legyen.
A végtelen jó tehát a legnagyobb jó.
Végtelenből szintén van pozitív és negatív.
\stopBek

\startBek[vegtelen_problema]
A végtelenben székelő személyes Isten elérhetetlen az ember számára.
A rekurzivitás alapján minden Istenben van, tehát a végtelen is Isten lakhelye kell legyen. %\resz{rekurziv1}
\stopBek

\startBek[pozitiv_szemelyiseg_az_egysegben]
Isten pozitív személyisége jó, fényes, bölcs, igazságos, szerető, szép, tiszta.
Ez megfelel a tökéletes isteni személyiség kritériumának.
\stopBek

\startBek[szemelyes_szekhely]
A személyes Isten székhelye tehát a pozitív egység és a végtelen, és ezáltal a teljes pozitívum.
Isten az egységben található meg az ember számára, ez tehát Isten lakhelyének a bejárata.
Dolgozószobája viszont a végtelen, ahova Ember be nem léphet.
Ez megfelel a mindenhatóságból levezetett állításnak, miszerint Isten egyszerre egy és végtelen.
\stopBek

\startBek[negativum_problema]
Ha a pozitívum a személyes Isten székhelye, akkor mi a negatívum és ki képvseli?
Negatívum nélkül nincs pozitívum, csak semlegesség van.
A semlegességben pedig mint láttuk, nincs személyiség.
\stopBek

\startBek[negativ_szemelyiseg_problema]
A negatív isteni személyiség szükségszerűen rossz, sötét, ostoba, igazságtalan, gyűlölő, csúf, és mocskos.
Úgy vélem, hogy az ostobaság kritériuma elegendő ahhoz, hogy a tökéletességi feltétellel ellentmondásba kerüljünk.
\stopBek

\startBek[negativum_az_emberben]
Istenen kívül az embernek van személyisége ami ráadásul gyakran vesz fel negatív jegyeket.
Ezek szerint a negatívum az ember sajátja.
Mégis, e nélkül a személyes Isten és a teremtett világ pozitív volta nem tudna megmutatkozni.
Az Ember számára kijelölt határok tehát rendkívül tágak: az egész isteni szféra sötét párja az ő felfedezésére vár.
A negatívumot az ember tárja fel; a személyes Isten közvetlenül nem, csak mint az ember része keresi fel azt.
Rendkívüli csábítás ez.
Az ember tehát minden elszenvedett rossz okozója.
És habár végtelen a feltárható negatívum, az ember halandósága, feledékenysége, valamint az isteni vonzerő megakadályozza azt, hogy túl messze jusson abban.
\stopBek

\startBek[negativum_nelkul]
Isten pedig azért engedi a rosszat érvényesülni, mert különben az embert a tudatlanságba, állati minőségbe kellene letaszítania.
Ellentétek nélkül minden visszazuhanna a végtelen semlegességbe.
A semlegességben nincs öröm, nincs furulyaszó és tánc, nincs kellemes társaság, nincs cél és nincs út.
A semlegességből nincs kitörési, felemelkedési lehetőség.
Ott Isten ugyan mindenkit maga mellett tudhatna, de senkit hozzá közel.
Többet ér az üres harmóniánál a számtalan viszontagságon átverekedő, a rosszban ezerszer megmártózó de később kimosakodó, milliárdból egy beérkezett Ember.
\stopBek

\startBek[isten_az_emberben_problema]
A személyes Isten emberi személyiségen keresztüli megközelítése ugyan sok mindent megmagyaráz, mégis kételyeket ébreszt.
Úgy tűnhet, hogy a személyes Isten önmagában nem, csak az ember által az emberben létezik.
Legyen a személyes Isten akár abszolút -- a rekurzivitás alapján szerintem az --, akár az emberi kollektív tudat része, vagy akár csak egy mém, az ember teendője akkor sem változik.
\stopBek

\startBek[tokeletes_ember_az_oseroben]
A tökéletes és nagybetűs Ember az emberek között a megtestesült Isten.
Ő az ember elképzelhető legjobbja, személye a legmagasabb rangú, isteni.
Az tökéletes Ember személyisége ugyanúgy a végtelenben látszik felsejleni, mint az abszolút személyes Istené, mégpedig azért, mert a nagybetűs Embert teljesen átjárja az Őserő, önmagát pedig átadta a őssodrásnak.
A személytelen és végtelen Őserő szerinti élet független a személyes Isten létezésétől.
\stopBek

\startBek[tokeletes_ember_imaja]
A tökéletes Ember tisztán lát, tud, hibát nem vétve éli békés mindennapjait.
Ő akit áthat az Őserő, tisztánlátása alapján imádkozik vagy nem imádkozik a személyes Istenhez.
\stopBek

\startBek[tokeletlen_ember_imaja]
Én viszont, aki még küzdök a démonjaimmal, amellett, hogy igyekszem igazodni az Őserőhöz, nagy megnyugvást lelek a személyes Istenben.
Don Juan azt tanítja, hogy cselekedjek mindent úgy, mintha az az utolsó földi csatám volna.
Próbáltam évekig sikertelenül: csak egyszer-egyszer értem el rövid időre a kívánt állapotot.
Aztán olvastam Szvámi Prabhupáda tanítását, miszerint folyamatosan vibráltassuk önmagunkban a személyes Isten nevét.
Ez nálam sokkal jobban működik.
A mélyen átélt ima közben az ember az Isten törvénye szerint cselekszik mint Isten áldozatos földi helytartója.
Isten törvénye pedig az Őserő sodrásában van.
Az ilyen cselekedet tehát magasabb minőségű mint utolsó földi csata, hiszen azon túl még áldozat is az Istennek, ami által derű tölti el az azt végző személyt.
\stopBek

\startBek[pozitiv_isteni_szemelyiseg_az_atya]
Most, hogy megindokoltam, hogy miért gondolom azt, hogy az Isten személyisége pozitív és tökéletes, hívhatom Őt Atyának.
\stopBek

\startBek[mindenhatosagi_kov_teremto_szemelyes]
A teremtő Atya a mindenséget az Ő személytelen formákából, az Őserőből hozta létre.
Ő tehát a mag mindenben, Ő az örök nemző, Ő a végső Atya.
A teremtés oka az Isten szeretete és a Sátán vágya lehetett.
\stopBek


%
%\startBek[mindenhatosagi_kov_teremto_szemelytelen]
%A mindenség viszont az Őserőből formálódott.
%Az Őserő tehát az áradó erő, amiből minden más keletkezik.
%Az Őserő tehát mindenben megvan, és ahol az Őserő, ott van az Isten is.
%\stopBek
%
%\startBek[mindenhatosagi_kov_elteto_szemelytelen]
%Az Őserő a fenntartója mindennek, az Őserő energiája élteti a mindenséget.
%Az Őserő tökéletesen semleges és semlegesen tökéletes.
%Ilyen a természet, a teremtő Isten munkája.
%\stopBek
%
%\startBek[mindenhatosagi_kov_elteto_szemelyes]
%Az Isten és a Sátán pedig gondoskodnak az emberről.
%Istentől a szeretet, Sátántól az érdek.
%Istentől a önuralom, Sátántól a vágy.
%Istentől a szintézis, Sátántól az analízis.
%\stopBek
%
%\startBek[mindenhatosagi_kov_hullajto]
%Az Atya a Halál.
%Ő a legbölcsebb kaszás: mindig akkor arat, amikor eljött az ideje.
%Az Őserő lüktetése az élet, és e lüktetésben a kiáradást visszatérés kell kövesse.
%\stopBek
%
%\startBek[mindenhatosagi_kov_magyarazat]
%Minden Istenben van és Isten is mindenben benne van.
%Az anyagi világ is Isten emanációja, a teremtmények pedig Isten által élnek.
%Minden, amire van fogalmunk, Isten legitim teremtménye, része, és áldozata, ugyanakkor magába is foglalja Istent.
%\stopBek
%

%
%\startBek[jo_rossz_semleges]
%Ha felismertük a jó és a rossz és ezekkel együtt a semleges fogalmát, akkor Isten egyszerre mindhárom.
%Isten tehát pozitív és negatív és semleges is egyszerre. \lasd{Sátán}
%\stopBek
%
%\startBek[jo_rossz_semleges_dim]
%Ahol a jó és a rossz vagy bármilyen más minőségi ellentétpár fokozatosan értelmezett, ott az végtelen szélességben és mélységben van meg Istenben.
%Ahol a jó és a rossznak vagy bármilyen más minőségi ellentétpárnak nincs fokozatossága, azaz vagy van vagy nincs az adott dolog, ott Istenben van is meg nincs is.
%Ez által emelkedik ki a végtelenből a pozitív és a negatív egység, valamint az origó, a semlegesség és a szintézis helye. %\lasd{szintezis<>analizis}
%\stopBek
%
%\startBek[jo_rossz_semleges_forma]
%Az egyetlen Atya székhelye az origó.
%Oda vezet az arany középút.
%
% az ellentétpároknak ott nem érvényesülnek.
%\stopBek
%
%
%
%
%
%
%
%
%
%
%
%
%
%\startBek[isten_megragadasa]
%Isten megragadható az Ő személyes és annak személytelen formáján keresztül.
%\stopBek
%
%
%Az előbbit Atyának vagy Istennek, Tennek, és Sátánnak nevezem, attól függően, hogy a pozitív, a semleges, vagy a negatív személyes formáról beszélek.
%Atyának pedig azért nevezem a Jóistent, mert én magam az ő fiaként tudom Őt megragadni miközben Istennek ehhez a személyes formájához húzok a leginkább.
%Isten személytelen formáját Őserőnek nevezem.
%Az ember jól teszi, ha Isten személyes és személytelen formáját egyaránt igyekszik a lehető legjobban megragadni.
%\stopBek
%
%
%
%
%
%
%
%A mindenhatósági axióma szerint tehát a semlegesség Isten egységében, míg a jó és a rossz Isten végtelenségében van.
%Hasonlóan, az éltetés az isteni egységben, míg a teremtés és a hullajtás az isteni végtelenségben vannak.
%Ugyanígy kell lennie minden ellentétpárra és az azok közti határra is: ez utóbbi az isteni egységben van, a kettősség pedig már végtelenség.
%Az egység mozdulatlan: nincs benne mozgástér.
%Ez az eredő, a forrás, az egy Isten otthona.
%Odakint a végtelen szabadság viszontagságos mezeje. \lasd{anyag}
%E mező közepén arany út vezet az Isten otthonába -- haza. \lasd{őshaza}
%\stopBek
\stopSzo
\stopcomponent
