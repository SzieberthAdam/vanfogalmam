\startcomponent isten
\product szavak
\startSzo[title=Isten,reference=szo:Isten]
\startBek[orok_es_mindenhato]
Hiszem, és ebből indulok ki: {\em Isten örök és mindenható.} \lasd{hit}
Mindig volt és mindig lesz.
Mindenre hat Isten; amire ne hatna, olyan nincsen.
\stopBek

\startBek[szeret_es_eltet]
Hogyan hat Isten?
Ezt pedig így hiszem: {\em szeret és éltet.}
Ha Isten örök akkor az éltetés is örök.
Teremtés és pusztítás nem más, mint változás.
Születés és halál szintén csak változás.
Minden élt már korábban és élni fog aztán is.
Él a kő is: más volt hajdan és más lesz majdan.
Isten az eredő, Isten a nyelő, ki- és belélegző.
Így lüktet minden, így ver a szívem.
Nincs is más semmi, csakis az Isten.
\stopBek

\startBek[formai]
Milyen az Isten?
Erre szó nincsen.
Így írja minden könyv és így mondja minden pap.
Értelemmel közelíteni, van annak értelme?
Isten egyes, kettős, hármas, vagy akárhányas?
Aki ilyet gondol, Istent nem látja mindenhol.
Isten része minden megnyilvánulás és megnemnyilvánulás.
A mindenható önmagára is hat, végtelen mélységek vannak.
Végtelen szférák, végtelen síkok, mindenütt isteni mintázatok.
Minden Istenben van és Isten is mindenben benne van.
Minden egészben van rész és minden részben ott az egész.
Örök a szeretet, de örök a változás: önmagára hatva még Isten is más.
Lehet jobb nagyítóm, lehet jobb távcsövem, mégis az azonos mintákat figyelem.
Minden nagyon hasonló, de végtelen a változat.
Lehetetlen ezek számba vétele véges idő alatt.
Mindben ott a közös, az Ős, az Ís, az Isten.
Nem kell ide eszköz, mert újat nem mutat.
Isten bennem és köröttem; magam elég vagyok, hogy elérjem.
Most a Földön vagyok, ember vagyok.
Bennem a szívem, bennem a lelkem, ezeket kell kövessem.
Kint Isten nagyságát mutatja a természet végtelen csodája.
A természeti törvényeket kell kövessem.
Elmerülni a békés rendbe, sodródni annak változásával, és itt-ott, mint jámbor gyermek, valamivel kedveskedni az Atyának, mást ne is akarjak.
Atyát mondtam, mert gyermeknek érzem magam.
Ő, aki mindörökké éltet engem, ki más volna nekem, mint a legdrágább Atyám?
Most a Földön vagyok, ember vagyok, küldetésben vagyok.
Ha én magam személy vagyok és Istenről tudok, Isten is személy.
A végtelen formát meg nem ragadhatom, Atyámat viszont könnyen imádhatom.
Fia vagyok viszont, nem a szolgája, mint derék fia váljak Atyám hasznos szolgálatára.
\stopBek

\startBek[viszonya]
Atyám teremt és Atyám pusztít.
Akkor vet és akkor arat, amikor az a legalkalmasabb.
Ő minden csapás mérője, minden kór okozója, sorsunk szabója.
Aki efölött kesereg, tudása még nálamnál is kevesebb.
A sors kijár, a cserebogár marad sárga cserebogár.
Így csodás a mindig változó megnyilvánulás.
Isten része az istentelenség: Isten kivonulhat de mégis ott marad.
Mert Isten része minden rossz és gonoszság ami valaha kiaknázatlanul állt.
Rendesen nem volt rossz, mert nem volt ki feltárja, és akkor sem lesz, ha mind aki bolygatja, e dolgát abbahagyja.
Tudva indulhatok az isteni fénybe, vagy törhetek tudatlanul a már feltárt sötétségbe.
Anyagom súlya lefelé húz ha értelmem fel nem emel.
Így küzd bennem anyag és értelem, és kezdetben sajnos az anyag van fölényben.
Testem a gépem, gépiesen működöm ameddig értelmem nem áll készen, hogy a kormányt átvegye, és lelkem átmenetileg a fény felé terelje.
Ezerszer is újra a test irányít, mert gyorsabb mint az értelem, és én ezerszer újra átveszem, hogy fordítva legyen.
Most pedig épphogy emelkedvén, vagy lassabban ereszkedvén látom, hogy sokan mily nagy buzgósággal törnek lefelé a mélybe, és viszik a maguk lángját elfojtani a sötétségbe.
Versengve keresik az újabb fertelmeket magukra véve a későbbi gyötrelmeket.
\stopBek


\stopSzo
\stopcomponent
